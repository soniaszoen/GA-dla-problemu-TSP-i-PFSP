% Options for packages loaded elsewhere
\PassOptionsToPackage{unicode}{hyperref}
\PassOptionsToPackage{hyphens}{url}
%
\documentclass[
]{article}
\usepackage{amsmath,amssymb}
\usepackage{iftex}
\ifPDFTeX
  \usepackage[T1]{fontenc}
  \usepackage[utf8]{inputenc}
  \usepackage{textcomp} % provide euro and other symbols
\else % if luatex or xetex
  \usepackage{unicode-math} % this also loads fontspec
  \defaultfontfeatures{Scale=MatchLowercase}
  \defaultfontfeatures[\rmfamily]{Ligatures=TeX,Scale=1}
\fi
\usepackage{lmodern}
\ifPDFTeX\else
  % xetex/luatex font selection
\fi
% Use upquote if available, for straight quotes in verbatim environments
\IfFileExists{upquote.sty}{\usepackage{upquote}}{}
\IfFileExists{microtype.sty}{% use microtype if available
  \usepackage[]{microtype}
  \UseMicrotypeSet[protrusion]{basicmath} % disable protrusion for tt fonts
}{}
\makeatletter
\@ifundefined{KOMAClassName}{% if non-KOMA class
  \IfFileExists{parskip.sty}{%
    \usepackage{parskip}
  }{% else
    \setlength{\parindent}{0pt}
    \setlength{\parskip}{6pt plus 2pt minus 1pt}}
}{% if KOMA class
  \KOMAoptions{parskip=half}}
\makeatother
\usepackage{xcolor}
\usepackage[margin=1in]{geometry}
\usepackage{graphicx}
\makeatletter
\newsavebox\pandoc@box
\newcommand*\pandocbounded[1]{% scales image to fit in text height/width
  \sbox\pandoc@box{#1}%
  \Gscale@div\@tempa{\textheight}{\dimexpr\ht\pandoc@box+\dp\pandoc@box\relax}%
  \Gscale@div\@tempb{\linewidth}{\wd\pandoc@box}%
  \ifdim\@tempb\p@<\@tempa\p@\let\@tempa\@tempb\fi% select the smaller of both
  \ifdim\@tempa\p@<\p@\scalebox{\@tempa}{\usebox\pandoc@box}%
  \else\usebox{\pandoc@box}%
  \fi%
}
% Set default figure placement to htbp
\def\fps@figure{htbp}
\makeatother
\setlength{\emergencystretch}{3em} % prevent overfull lines
\providecommand{\tightlist}{%
  \setlength{\itemsep}{0pt}\setlength{\parskip}{0pt}}
\setcounter{secnumdepth}{-\maxdimen} % remove section numbering
\usepackage{bookmark}
\IfFileExists{xurl.sty}{\usepackage{xurl}}{} % add URL line breaks if available
\urlstyle{same}
\hypersetup{
  pdftitle={Algorytm genetyczny dla problemu komiwojażera (TSP)},
  hidelinks,
  pdfcreator={LaTeX via pandoc}}

\title{Algorytm genetyczny dla problemu komiwojażera (TSP)}
\author{}
\date{\vspace{-2.5em}2025-10-24}

\begin{document}
\maketitle

\section{Problem komiwojażera TSP}\label{problem-komiwojaux17cera-tsp}

\subsection{Założenia dotyczące problemu
TSP}\label{zaux142oux17cenia-dotyczux105ce-problemu-tsp}

Rozważany problem to klasyczny Problem Komiwojażera (Traveling Salesman
Problem, TSP), w którym należy znaleźć najkrótszą możliwą trasę
odwiedzającą każde miasto dokładnie raz i wracającą do punktu
startowego.

\begin{itemize}
\item
  Odległości między miastami są znane i symetryczne, czyli droga z
  miasta A do B ma tę samą długość co z B do A.
\item
  Wartości w macierzy odległości są nieujemne i odpowiadają rzeczywistym
  dystansom.
\item
  Każde miasto musi być odwiedzone dokładnie jeden raz, a trasa jest
  cykliczna (powrót do punktu początkowego).
\end{itemize}

\section{Funkcja celu}\label{funkcja-celu}

Główną metryką jest długość trasy (im mniejsza, tym lepsza).

\section{Dane}\label{dane}

Pliki wejściowe (macierze odległości):

\begin{itemize}
\item
  Dane\_TSP\_127.xlsx - 127 miast
\item
  Dane\_TSP\_76.xlsx - 76 miast
\item
  Dane\_TSP\_48.xlsx - 48 miast
\end{itemize}

Algorytm wykorzystuje bezpośrednio tę macierz do oceny tras.

\section{Wstęp}\label{wstux119p}

Badamy algorytm genetyczny (GA) rozwiązujący problem komiwojażera (TSP).
Celem jest porównanie wpływu kluczowych parametrów algorytmu na jakość
znalezionych tras, przyjmując podejście one-factor-at-a-time (OFAT)
czyli zmieniamy kolejno po jednym parametrze, a pozostałe pozostają
stałe (\texttt{baseline}). Wyniki zapisujemy do pliku Excela i
analizujemy wykresy.

\section{Środowisko i ustawienia
powtarzalności}\label{ux15brodowisko-i-ustawienia-powtarzalnoux15bci}

\begin{itemize}
\tightlist
\item
  \textbf{Globalny seed}: \texttt{SEED\ =\ 42} jest ustawiany na
  początku działania skryptu.
\item
  \textbf{Seed dla uruchomienia}: Dla każdego pojedynczego uruchomienia
  algorytmu, seed jest ustawiany deterministycznie
  (\texttt{SEED\ +\ run\_id}), co pozwala na dokładne odtworzenie
  każdego eksperymentu.
\end{itemize}

\section{Założenia dotyczące algorytmu
genetycznego}\label{zaux142oux17cenia-dotyczux105ce-algorytmu-genetycznego}

Algorytm genetyczny jest metodą metaheurystyczną czyli nie gwarantuje
znalezienia rozwiązania optymalnego, ale pozwala znaleźć rozwiązania
bliskie optymalnym w rozsądnym czasie.

Rozwiązania (osobniki) są reprezentowane jako permutacje miast -
kolejność odwiedzin miast określa trasę.

Populacja osobników ewoluuje przez wiele pokoleń (n\_gen) w kierunku
coraz krótszych tras.

\section{Opis parametrów}\label{opis-parametruxf3w}

\begin{itemize}
\item
  n\_pop oznacza rozmiar populacji; populacja początkowa jest tworzona
  losowo, a potem każda generacja jest w pełni zastępowana przez
  potomków + ewentualnie elity
\item
  n\_gen oznacza liczbę generacji; im więcej generacji, tym dłużej trwa
  ewolucja i algorytm ma więcej szans na poprawę wyniku ale zbyt wiele
  generacji może nie przynosić dalszych korzyści
\item
  p\_mut czyli prawdopodobieństwo mutacji, pomaga uciekać z lokalnych
  minimów; szansa, że po utworzeniu dziecka, jego trasa zostanie losowo
  zmieniona;w tym przypadku polega na zamianie (swap) dwóch miast w
  trasie; nie gwarantuje poprawy rozwiązania; (!nie należy mylić mutacji
  z ulepszaniem local search (ruchami sąsiedzkimi) które działają już po
  mutacji i próbują celowo poprawić trasę, a nie losowo ją zmienić)
\item
  p\_cx czyli prawdopodobieństwo krzyżowania, częstotliwość rekombinacji
  rodziców; to szansa, że z dwóch rodziców powstanie nowy potomek, który
  łączy ich cechy; jeśli losowanie się nie powiedzie, potomek jest po
  prostu kopią jednego rodzica;
\item
  selection\_method czyli metody selekcji; decyduje kto zostanie wybrany
  jako rodzic do stworzenia potomka, wszystkie metody dążą do tego, by
  lepsi mieli większą szansę na rozmnożenie, ale każda robi to trochę
  inaczej:
\end{itemize}

\begin{enumerate}
\def\labelenumi{\arabic{enumi})}
\item
  tournament (selekcja turniejowa) - losuje k osobników z populacji, z
  tych k wybiera tego o najkrótszej trasie (najlepszy), pozwala
  kontrolować siłę selekcji przez tournament\_k; im większe k, tym
  silniejsza selekcja (bo częściej wygra najlepszy z losowanych)
\item
  roulette (selekcja ruletkowa) - każdy osobnik dostaje „wagę'' na
  ruletce proporcjonalne do swojej jakości (u nas odwrotnie do długości
  trasy); daje większe szanse dobrym trasom, ale też pozwala czasem
  „słabszym'' wejść do gry (bo utrzymuje różnorodność)
\item
  rank (selekcja rankingowa) - najlepszy osobnik dostaje najwyższy
  „ranking'', najgorszy najniższy; szansa na wybór zależy od pozycji w
  rankingu, a nie bezpośrednio od długości trasy.
\end{enumerate}

\begin{itemize}
\tightlist
\item
  crossover\_method czyli metody krzyżowania, różne sposoby zachowania
  porządku i fragmentów rodziców w permutacji, czyli w jaki sposób
  łączone są ich geny (kolejności miast):
\end{itemize}

\begin{enumerate}
\def\labelenumi{\arabic{enumi})}
\item
  PMX (Partially Mapped Crossover) - wybierany jest losowy fragment (np.
  4-8 miast) z pierwszego rodzica i kopiowany do dziecka, reszta miast
  uzupełniana jest na podstawie drugiego rodzica, zgodnie z mapowaniem
  między fragmentami.
\item
  OX (Order Crossover) - fragment z jednego rodzica zostaje zachowany,
  pozostałe miasta są wypełniane w kolejności ich występowania w drugim
  rodzicu; dziecko dziedziczy „kolejność odwiedzin'', a niekoniecznie
  dokładne pozycje.
\item
  CX (Cycle Crossover) - tworzy cykle powiązań między genami rodziców
  (np. miasto A u pierwszego jest w tym samym miejscu co B u drugiego),
  naprzemiennie bierze cykle z rodzica 1 i 2.
\end{enumerate}

\begin{itemize}
\tightlist
\item
  local\_search\_tries: liczba prób lokalnego ulepszenia; czyli liczba
  ruchów sąsiedzkich, które są losowo testowane na każdym dziecku, by
  sprawdzić, czy da się poprawić trasę:
\end{itemize}

\begin{enumerate}
\def\labelenumi{\arabic{enumi})}
\item
  swap - zamiana dwóch miast miejscami,
\item
  2-opt - odwrócenie fragmentu trasy,
\item
  insertion - wyjęcie miasta i wstawienie go gdzie indziej
\end{enumerate}

! Uwaga ! Zamiast testować ruchy swap, insertion i two\_opt osobno,
funkcja local\_improve losowo wybiera jeden z nich w każdej próbie.

\begin{itemize}
\tightlist
\item
  elite (elitaryzm) oznacza liczbę najlepszych osobników kopiowanych bez
  zmian; chroni najlepsze rozwiązania, ale zbyt duża wartość elite
  mogłaby zmniejszyć różnorodność.
\end{itemize}

\section{Usprawnienie mechanizm „wnuka''
(grandchild)}\label{usprawnienie-mechanizm-wnuka-grandchild}

Mechanizm „wnuka'' (grandchild) to proste rozszerzenie standardowego
schematu rozmnażania w algorytmie genetycznym. Po wygenerowaniu potomka
(child) z pary rodziców istnieje pewne prawdopodobieństwo
p\_make\_grandchild, że z potomka powstanie dodatkowy twór - „wnuk''
(grandchild). Wnuk powstaje przez zastosowanie kolejnego, krótkiego
operatora naprawczego (np. prostej mutacji lub szybkiego lokalnego
ulepszania). Do populacji przyjmowany jest lepszy z pary (child vs
grandchild). Intuicyjnie, zamiast przyjmować bezwarunkowo pojedynczego
potomka, sprawdzamy jedną dodatkową opcję „bliźniaczą'' i zachowujemy
tę, która daje krótszą trasę.

Parametry:

\begin{itemize}
\item
  p\_make\_grandchild (= {[}0,1{]}) - prawdopodobieństwo wygenerowania
  wnuka dla każdego potomka,
\item
  grandchild\_method = \{`mutate', `local\_improve'\} - sposób
  wytworzenia wnuka: prosta mutacja (swap) lub krótki lokalny search
  (np. local\_improve z ruchem two\_opt albo mieszany),
\item
  grandchild\_local\_tries - liczba prób dla lokalnego ulepszania wnuka
  (jeśli None, używany jest parametr local\_search\_tries).
\end{itemize}

Mechanizm wnuka daje prosty kompromis:

\begin{itemize}
\item
  zwiększa szansę na znalezienie natychmiastowej ulepszonej wersji
  potomka (przy niskim koszcie - jedno wywołanie mutacji lub krótkiego
  local search);
\item
  jest prosty do kontrolowania przez pojedynczy parametr
  p\_make\_grandchild.
\end{itemize}

Typowe efekty obserwowane w eksperymentach:

\begin{itemize}
\item
  umiarkowane zwiększenie jakości końcowych rozwiązań (niższe średnie i
  lepsze mediany),
\item
  niewielki wzrost wariancji (jeśli p\_make\_grandchild zbyt duże),
\end{itemize}

Jednak po analizie okazuje się, że żaden z najlepszych wyników nie miał
włączonego mechanizmu wnuka.

\subsection{Baseline (parametry
domyślne)}\label{baseline-parametry-domyux15blne}

baseline = \{ - `n\_pop': 100, - `n\_gen': 300, - `p\_mut': 0.03, -
`p\_cx' : 0.9, - `selection\_method': `tournament', -
`crossover\_method': `pmx', - `local\_search\_tries': 5, -
`local\_moves': {[}`swap',`two\_opt',`insertion'{]}, - `elite': 3, -
`tournament\_k': 3, - `p\_make\_grandchild': 0.0, -
`grandchild\_method': `mutate', - `grandchild\_local\_tries': None \}

\subsection{Parametry testowane (OFAT)}\label{parametry-testowane-ofat}

Dla każdego parametru testujemy wartości:

\begin{itemize}
\item
  n\_gen : {[}100, 300, 600, 1200{]}
\item
  n\_pop : {[}50, 100, 200, 400{]}
\item
  p\_mut : {[}0.01, 0.03, 0.07, 0.15{]}
\item
  p\_cx : {[}0.6, 0.75, 0.9, 1.0{]}
\item
  selection\_method : {[}`tournament', `roulette', `rank'{]}
  (kategorialnie)
\item
  crossover\_method : {[}`pmx', `ox', `cx'{]} (kategorialnie)
\item
  local\_search\_tries : {[}0, 5, 20, 50{]}
\item
  crossover\_method`: {[}`pmx', `ox', `cx'{]},
\item
  selection\_method`: {[}`tournament', `roulette', `rank'{]}
\end{itemize}

Dla każdej konfiguracji powtarzamy uruchomienie repetitions razy 10.

\section{Dokładna analiza
wyników}\label{dokux142adna-analiza-wynikuxf3w}

\subsection{Dla 48 miast}\label{dla-48-miast}

\subsubsection{parametr elite}\label{parametr-elite}

\includegraphics[width=0.32\linewidth,height=\textheight,keepaspectratio]{results_excel/plots/Dane_TSP_48/elite/boxplot_best_length.png}
\includegraphics[width=0.32\linewidth,height=\textheight,keepaspectratio]{results_excel/plots/Dane_TSP_48/elite/mean_std.png}
\includegraphics[width=0.32\linewidth,height=\textheight,keepaspectratio]{results_excel/plots/Dane_TSP_48/elite/time_vs_best_scatter.png}

Wykres pudełkowy pokazuje, jak zmieniała się długość najlepszej trasy w
zależności od liczby osobników elity. Dla wartości elite = 3 (na
wykresie pudełka nie są po kolei) długości tras są mniejsze i bardziej
stabilne niż dla pozostałych wartości. Dla zbyt dużej elity, wyniki są
bardziej rozproszone, co oznacza większą losowość i gorszą
powtarzalność. Możemy stwierdzić, że umiarkowana wartość elity (około
2-3) daje najlepszy kompromis między jakością a stabilnością wyników. Na
drugim wykresie przedstawiamy średnią długość najlepszej trasy oraz
odchylenie standardowe dla różnych wartości elity. Najniższa średnia
wartość występuje przy elite = 3, więc mamy potwierdzenie, że ta
konfiguracja pozwala osiągać najlepsze rezultaty. Dla większej elity,
widać wyraźnie większe odchylenie standardowe, czyli mniejszą stabilność
działania algorytmu. Zbyt duża liczba elitarnych osobników może
utrudniać znalezienie optymalnej trasy.

\subsubsection{parametr n\_gen}\label{parametr-n_gen}

\includegraphics[width=0.32\linewidth,height=\textheight,keepaspectratio]{results_excel/plots/Dane_TSP_48/n_gen/boxplot_best_length.png}
\includegraphics[width=0.32\linewidth,height=\textheight,keepaspectratio]{results_excel/plots/Dane_TSP_48/n_gen/mean_std.png}
\includegraphics[width=0.32\linewidth,height=\textheight,keepaspectratio]{results_excel/plots/Dane_TSP_48/n_gen/time_vs_best_scatter.png}

Wraz ze wzrostem liczby generacji obserwujemy wyraźną poprawę jakości
rozwiązań czyli mniejszą długość najlepszej trasy. Na boxplocie widać,
że przy małej liczbie generacji wyniki są gorsze i różne, natomiast przy
300-600 generacjach rozrzut jest mniejszy, a wartości lepsze. Średnie
wartości na wykresie 2. potwierdzają, że około 300-600 generacji daje
najniższe średnie długości tras. Dla 1000 generacji wynik nie poprawia
się znacząco, natomiast czas obliczeń rośnie, co widać na wykresie 3.
Zatem zwiększanie liczby generacji po pewnym progu przynosi coraz
mniejsze korzyści, a tylko wydłuża działanie algorytmu. Optymalny
kompromis między jakością a czasem obliczeń osiągamy przy około 300-600
generacjach.

\subsubsection{parametr n\_pop}\label{parametr-n_pop}

\includegraphics[width=0.32\linewidth,height=\textheight,keepaspectratio]{results_excel/plots/Dane_TSP_48/n_pop/boxplot_best_length.png}
\includegraphics[width=0.32\linewidth,height=\textheight,keepaspectratio]{results_excel/plots/Dane_TSP_48/n_pop/mean_std.png}
\includegraphics[width=0.32\linewidth,height=\textheight,keepaspectratio]{results_excel/plots/Dane_TSP_48/n_pop/time_vs_best_scatter.png}
Parametr liczby osobników w populacji wpływa na jakość rozwiązań i czas
działania algorytmu. Na wykresach widać, że dla małych populacji wyniki
są bardziej zmienne, ale średnia długość tras jest zbliżona do
najlepszych przypadków. Przy populacji 200 uzyskano najniższe wartości
długości tras. Dla większych populacji (jak 400), poprawa jakości nie
jest już zauważalna, natomiast czas obliczeń znacznie wzrasta. Z wykresu
3. widać, że im większa populacja, tym dłuższy czas działania, ale nie
zawsze idzie to w parze z lepszym wynikiem. Optymalna liczebność
populacji w tym przypadku to około 200 osobników.

\subsubsection{parametr p\_cx}\label{parametr-p_cx}

\includegraphics[width=0.32\linewidth,height=\textheight,keepaspectratio]{results_excel/plots/Dane_TSP_48/p_cx/boxplot_best_length.png}
\includegraphics[width=0.32\linewidth,height=\textheight,keepaspectratio]{results_excel/plots/Dane_TSP_48/p_cx/mean_std.png}
\includegraphics[width=0.32\linewidth,height=\textheight,keepaspectratio]{results_excel/plots/Dane_TSP_48/p_cx/time_vs_best_scatter.png}

Prawdopodobieństwo krzyżowania wpływa na intensywność wymiany materiału
genetycznego między osobnikami. Na wykresach widać, że wartości
pośrednie prowadzą do uzyskania najkrótszych tras oraz stosunkowo
niskiego rozrzutu wyników. Przy zbyt niskim p\_cx (0.6) wyniki są gorsze
i bardziej niestabilne.Na wykresie czasu widać, że różnice w czasie
działania są niewielkie, więc ten parametr nie wpływa istotnie na
wydajność. Można więc uznać, że optymalny zakres p\_cx to około
0,75-0,9, gdzie algorytm osiąga najlepsze wyniki jakościowe.

\subsubsection{parametr p\_mut}\label{parametr-p_mut}

\includegraphics[width=0.32\linewidth,height=\textheight,keepaspectratio]{results_excel/plots/Dane_TSP_48/p_mut/boxplot_best_length.png}
\includegraphics[width=0.32\linewidth,height=\textheight,keepaspectratio]{results_excel/plots/Dane_TSP_48/p_mut/mean_std.png}
\includegraphics[width=0.32\linewidth,height=\textheight,keepaspectratio]{results_excel/plots/Dane_TSP_48/p_mut/time_vs_best_scatter.png}

Zbyt małe wartości p\_mut (0,01) powodowały przedwczesną zbieżność i
gorsze wyniki. Wraz ze wzrostem do 0,03 i , poprawiała się zarówno
średnia, jak i stabilność rezultatów. Przy p\_mut = 0,15 obserwowano już
spadek jakości, zbyt częste mutacje zaburzały proces ewolucji. Najlepszy
balans uzyskano przy prawdopodobieństwie wynoszącym około 0,05-0,07.

\subsubsection{parametr
p\_make\_grandchild}\label{parametr-p_make_grandchild}

\includegraphics[width=0.32\linewidth,height=\textheight,keepaspectratio]{results_excel/plots/Dane_TSP_48/p_make_grandchild/boxplot_best_length.png}
\includegraphics[width=0.32\linewidth,height=\textheight,keepaspectratio]{results_excel/plots/Dane_TSP_48/p_make_grandchild/mean_std.png}
\includegraphics[width=0.32\linewidth,height=\textheight,keepaspectratio]{results_excel/plots/Dane_TSP_48/p_make_grandchild/time_vs_best_scatter.png}

Większy udział „wnuków'' w populacji (czyli osobników tworzonych z
dodatkowej rekombinacji) prowadził do nieco lepszych wyników, ale też
większej zmienności. Dla p\_make\_grandchild = 0,5 wyniki były
najbardziej stabilne, a dalsze zwiększanie tej wartości nie dawało już
korzyści. Niskie wartości parametru skutkowały szybkim utknięciem w
lokalnych minimach. Optymalny zakres to 0,4-0,5.

\subsubsection{parametr
local\_search\_tries}\label{parametr-local_search_tries}

\includegraphics[width=0.32\linewidth,height=\textheight,keepaspectratio]{results_excel/plots/Dane_TSP_48/local_search_tries/boxplot_best_length.png}
\includegraphics[width=0.32\linewidth,height=\textheight,keepaspectratio]{results_excel/plots/Dane_TSP_48/local_search_tries/mean_std.png}
\includegraphics[width=0.32\linewidth,height=\textheight,keepaspectratio]{results_excel/plots/Dane_TSP_48/local_search_tries/time_vs_best_scatter.png}

Zwiększenie liczby prób lokalnego przeszukiwania z 0 do 5 powoduje
drastyczną poprawę wyników (średnia długość trasy spada z bardzo
wysokich wartości) i znacznie zmniejsza rozrzut wyników. Dalsze
zwiększenie prób do 10 i 20 daje już niewielkie dalsze korzyści. Scatter
time vs best\_length pokazuje, że konfiguracje z większą liczbą prób
dają lepsze długości ale wymagają więcej czasu.

\subsubsection{parametr
crossover\_method}\label{parametr-crossover_method}

\includegraphics[width=0.32\linewidth,height=\textheight,keepaspectratio]{results_excel/plots/Dane_TSP_48/crossover_method/boxplot_best_length.png}\includegraphics[width=0.32\linewidth,height=\textheight,keepaspectratio]{results_excel/plots/Dane_TSP_48/crossover_method/mean_std.png}
\includegraphics[width=0.32\linewidth,height=\textheight,keepaspectratio]{results_excel/plots/Dane_TSP_48/crossover_method/time_vs_best_scatter.png}

PMX pozostaje najbardziej stabilny.

\subsubsection{parametr
selection\_method}\label{parametr-selection_method}

\includegraphics[width=0.32\linewidth,height=\textheight,keepaspectratio]{results_excel/plots/Dane_TSP_48/selection_method/boxplot_best_length.png}\includegraphics[width=0.32\linewidth,height=\textheight,keepaspectratio]{results_excel/plots/Dane_TSP_48/selection_method/mean_std.png}
\includegraphics[width=0.32\linewidth,height=\textheight,keepaspectratio]{results_excel/plots/Dane_TSP_48/selection_method/time_vs_best_scatter.png}
Selekcja poprzez turniej zdaje się być najlepszym wyborem.

\subsection{Dane 76 miast}\label{dane-76-miast}

\subsubsection{parametr elite}\label{parametr-elite-1}

\includegraphics[width=0.32\linewidth,height=\textheight,keepaspectratio]{results_excel/plots/Dane_TSP_76/elite/boxplot_best_length.png}
\includegraphics[width=0.32\linewidth,height=\textheight,keepaspectratio]{results_excel/plots/Dane_TSP_76/elite/mean_std.png}
\includegraphics[width=0.32\linewidth,height=\textheight,keepaspectratio]{results_excel/plots/Dane_TSP_76/elite/time_vs_best_scatter.png}
Rozmiar elity znacząco wpływała na stabilność wyników. Dla elite = 1-3
trasy były krótsze i bardziej powtarzalne, natomiast przy elite = 10
pojawiało się więcej outlierów. Zbyt duża elita ograniczała różnorodność
populacji, przez co algorytm szybciej tracił zdolność eksploracji.
Najbardziej zrównoważone rezultaty uzyskano przy elite = 2.

\subsubsection{parametr n\_gen}\label{parametr-n_gen-1}

\includegraphics[width=0.32\linewidth,height=\textheight,keepaspectratio]{results_excel/plots/Dane_TSP_76/n_gen/boxplot_best_length.png}
\includegraphics[width=0.32\linewidth,height=\textheight,keepaspectratio]{results_excel/plots/Dane_TSP_76/n_gen/mean_std.png}
\includegraphics[width=0.32\linewidth,height=\textheight,keepaspectratio]{results_excel/plots/Dane_TSP_76/n_gen/time_vs_best_scatter.png}

Wraz ze wzrostem liczby generacji poprawiała się jakość rozwiązań, choć
przy n\_gen \textgreater{} 600 poprawa była już minimalna. Dla n\_gen =
300-600 algorytm osiągał stabilne i powtarzalne wyniki. Wykres czasu
pokazuje, że wzrost liczby generacji liniowo zwiększał koszt obliczeń,
więc rekomendowany kompromis to n\_gen = 600.

\subsubsection{parametr n\_pop}\label{parametr-n_pop-1}

\includegraphics[width=0.32\linewidth,height=\textheight,keepaspectratio]{results_excel/plots/Dane_TSP_76/n_pop/boxplot_best_length.png}
\includegraphics[width=0.32\linewidth,height=\textheight,keepaspectratio]{results_excel/plots/Dane_TSP_76/n_pop/mean_std.png}
\includegraphics[width=0.32\linewidth,height=\textheight,keepaspectratio]{results_excel/plots/Dane_TSP_76/n_pop/time_vs_best_scatter.png}

Zwiększenie rozmiaru populacji poprawiało medianę długości tras aż do
n\_pop = 200. Dla n\_pop = 400 efekty się wyrównywały, co sugeruje
osiągnięcie wystarczającej różnorodności. Mniejsze populacje
charakteryzowały się większą wariancją wyników. Dlatego n\_pop = 200
daje najlepszy stosunek jakości do czasu.

\subsubsection{parametr p\_cx}\label{parametr-p_cx-1}

\includegraphics[width=0.32\linewidth,height=\textheight,keepaspectratio]{results_excel/plots/Dane_TSP_76/p_cx/boxplot_best_length.png}
\includegraphics[width=0.32\linewidth,height=\textheight,keepaspectratio]{results_excel/plots/Dane_TSP_76/p_cx/mean_std.png}
\includegraphics[width=0.32\linewidth,height=\textheight,keepaspectratio]{results_excel/plots/Dane_TSP_76/p_cx/time_vs_best_scatter.png}

Najlepsze wyniki uzyskano przy p\_cx = 0.9, co potwierdza skuteczność
częstego krzyżowania dla średniej wielkości zbioru. Dla mniejszych
wartości (0.6-0.75) trasy były dłuższe, a rozrzut wyników większy. p\_cx
= 1.0 nie przyniosło dalszej poprawy, sugerując nasycenie efektu.

\subsubsection{parametr p\_mut}\label{parametr-p_mut-1}

\includegraphics[width=0.32\linewidth,height=\textheight,keepaspectratio]{results_excel/plots/Dane_TSP_76/p_mut/boxplot_best_length.png}
\includegraphics[width=0.32\linewidth,height=\textheight,keepaspectratio]{results_excel/plots/Dane_TSP_76/p_mut/mean_std.png}
\includegraphics[width=0.32\linewidth,height=\textheight,keepaspectratio]{results_excel/plots/Dane_TSP_76/p_mut/time_vs_best_scatter.png}

Średni poziom mutacji (0.05-0.07) okazał się najkorzystniejszy. Zbyt
niskie wartości prowadziły do stagnacji, a zbyt wysokie (0.15)
powodowały losowe skoki i niestabilność wyników. Na wykresie widać, że
wariancja była najmniejsza przy p\_mut = 0.05, co wskazuje na dobrą
równowagę eksploracji i eksploatacji.

\subsubsection{parametr
p\_make\_grandchild}\label{parametr-p_make_grandchild-1}

\includegraphics[width=0.32\linewidth,height=\textheight,keepaspectratio]{results_excel/plots/Dane_TSP_76/p_make_grandchild/boxplot_best_length.png}
\includegraphics[width=0.32\linewidth,height=\textheight,keepaspectratio]{results_excel/plots/Dane_TSP_76/p_make_grandchild/mean_std.png}
\includegraphics[width=0.32\linewidth,height=\textheight,keepaspectratio]{results_excel/plots/Dane_TSP_76/p_make_grandchild/time_vs_best_scatter.png}

Parametr ten miał umiarkowany wpływ na wyniki, jednak dla średnich
wartości (ok. 0,4) uzyskano najlepsze trasy. Niskie wartości powodowały
utratę różnorodności, a zbyt wysokie - nadmierną losowość. Na wykresie
mean\_std widać, że odchylenie standardowe było najmniejsze właśnie w
tym zakresie.

\subsubsection{parametr
local\_search\_tries}\label{parametr-local_search_tries-1}

\includegraphics[width=0.32\linewidth,height=\textheight,keepaspectratio]{results_excel/plots/Dane_TSP_76/local_search_tries/boxplot_best_length.png}
\includegraphics[width=0.32\linewidth,height=\textheight,keepaspectratio]{results_excel/plots/Dane_TSP_76/local_search_tries/mean_std.png}
\includegraphics[width=0.32\linewidth,height=\textheight,keepaspectratio]{results_excel/plots/Dane_TSP_76/local_search_tries/time_vs_best_scatter.png}

Podobnie jak w mniejszym zbiorze, wyłączenie lokalnego przeszukiwania
skutkuje znacznie gorszymi trasami, natomiast przejście do wartości 5-20
daje duży skok jakości (średnie spadają). Wartości 10-20 nie dają już
proporcjonalnie większej poprawy jakości. Wnioskiem jest, że dla tej
instancji także warto włączyć krótkie lokalne ulepszanie.

\subsubsection{parametr
crossover\_method}\label{parametr-crossover_method-1}

\includegraphics[width=0.32\linewidth,height=\textheight,keepaspectratio]{results_excel/plots/Dane_TSP_76/crossover_method/boxplot_best_length.png}\includegraphics[width=0.32\linewidth,height=\textheight,keepaspectratio]{results_excel/plots/Dane_TSP_76/crossover_method/mean_std.png}
\includegraphics[width=0.32\linewidth,height=\textheight,keepaspectratio]{results_excel/plots/Dane_TSP_76/crossover_method/time_vs_best_scatter.png}

PMX daje najlepsze wyniki.

\subsubsection{parametr
selection\_method}\label{parametr-selection_method-1}

\includegraphics[width=0.32\linewidth,height=\textheight,keepaspectratio]{results_excel/plots/Dane_TSP_76/selection_method/boxplot_best_length.png}\includegraphics[width=0.32\linewidth,height=\textheight,keepaspectratio]{results_excel/plots/Dane_TSP_76/selection_method/mean_std.png}
\includegraphics[width=0.32\linewidth,height=\textheight,keepaspectratio]{results_excel/plots/Dane_TSP_76/selection_method/time_vs_best_scatter.png}
Tutaj również optymalnym wyborem jest selekcja poprzez turniej.

\subsection{Dane 127 miast}\label{dane-127-miast}

\subsubsection{parametr elite}\label{parametr-elite-2}

\includegraphics[width=0.32\linewidth,height=\textheight,keepaspectratio]{results_excel/plots/Dane_TSP_127/elite/boxplot_best_length.png}
\includegraphics[width=0.32\linewidth,height=\textheight,keepaspectratio]{results_excel/plots/Dane_TSP_127/elite/mean_std.png}
\includegraphics[width=0.32\linewidth,height=\textheight,keepaspectratio]{results_excel/plots/Dane_TSP_127/elite/time_vs_best_scatter.png}

Dla największego zestawu danych umiarkowana elita (2-3 osobniki) była
kluczowa dla stabilności. Przy braku elity wyniki były niestabilne, a
najlepsze trasy często ginęły między pokoleniami. Natomiast zbyt duża
elita (np. 10) powodowała stagnację i brak poprawy po kilku generacjach.
Optymalna wartość to 3.

\subsubsection{parametr n\_gen}\label{parametr-n_gen-2}

\includegraphics[width=0.32\linewidth,height=\textheight,keepaspectratio]{results_excel/plots/Dane_TSP_127/n_gen/boxplot_best_length.png}
\includegraphics[width=0.32\linewidth,height=\textheight,keepaspectratio]{results_excel/plots/Dane_TSP_127/n_gen/mean_std.png}
\includegraphics[width=0.32\linewidth,height=\textheight,keepaspectratio]{results_excel/plots/Dane_TSP_127/n_gen/time_vs_best_scatter.png}

Zwiększanie liczby generacji przynosiło wyraźną poprawę jakości
rozwiązań aż do poziomu 1200. Wykresy pokazują, że dla n\_gen
\textgreater{} 600 algorytm wciąż znajduje lepsze rozwiązania, choć z
malejącym tempem przyrostu jakości. Czas obliczeń rośnie jednak
proporcjonalnie, więc kompromisowy wybór to n\_gen = 600-900.

\subsubsection{parametr n\_pop}\label{parametr-n_pop-2}

\includegraphics[width=0.32\linewidth,height=\textheight,keepaspectratio]{results_excel/plots/Dane_TSP_127/n_pop/boxplot_best_length.png}
\includegraphics[width=0.32\linewidth,height=\textheight,keepaspectratio]{results_excel/plots/Dane_TSP_127/n_pop/mean_std.png}
\includegraphics[width=0.32\linewidth,height=\textheight,keepaspectratio]{results_excel/plots/Dane_TSP_127/n_pop/time_vs_best_scatter.png}

Dla dużej instancji największy wpływ miał rozmiar populacji. n\_pop =
200 zapewniał wyraźnie lepsze wyniki niż 50 czy 100, a dalsze
zwiększanie do 400 poprawiało medianę już tylko nieznacznie.
Jednocześnie czas wzrastał ponad dwukrotnie, dlatego n\_pop = 200 uznano
za najbardziej efektywny kompromis.

\subsubsection{parametr p\_cx}\label{parametr-p_cx-2}

\includegraphics[width=0.32\linewidth,height=\textheight,keepaspectratio]{results_excel/plots/Dane_TSP_127/p_cx/boxplot_best_length.png}
\includegraphics[width=0.32\linewidth,height=\textheight,keepaspectratio]{results_excel/plots/Dane_TSP_127/p_cx/mean_std.png}
\includegraphics[width=0.32\linewidth,height=\textheight,keepaspectratio]{results_excel/plots/Dane_TSP_127/p_cx/time_vs_best_scatter.png}

Najniższe długości tras uzyskano przy p\_cx = 0.9. Dla mniejszych
wartości poprawa była wolniejsza, a dla p\_cx = 1.0 - zauważalnie
większa wariancja. Wynika to z tego, że całkowite krzyżowanie wszystkich
par rodziców zwiększa losowość populacji. Optymalny zakres to p\_cx =
0.85-0.9.

\subsubsection{parametr p\_mut}\label{parametr-p_mut-2}

\includegraphics[width=0.32\linewidth,height=\textheight,keepaspectratio]{results_excel/plots/Dane_TSP_127/p_mut/boxplot_best_length.png}
\includegraphics[width=0.32\linewidth,height=\textheight,keepaspectratio]{results_excel/plots/Dane_TSP_127/p_mut/mean_std.png}
\includegraphics[width=0.32\linewidth,height=\textheight,keepaspectratio]{results_excel/plots/Dane_TSP_127/p_mut/time_vs_best_scatter.png}

Dla największej liczby miast mutacja miała silniejszy wpływ. Przy p\_mut
= 0.01 wyniki były bardzo słabe, a przy p\_mut = 0.15 - niestabilne.
Najlepsze rezultaty pojawiły się przy p\_mut = 0.07, gdzie algorytm
utrzymywał równowagę między eksploracją a zbieżnością.

\subsubsection{parametr
p\_make\_grandchild}\label{parametr-p_make_grandchild-2}

\includegraphics[width=0.32\linewidth,height=\textheight,keepaspectratio]{results_excel/plots/Dane_TSP_127/p_make_grandchild/boxplot_best_length.png}
\includegraphics[width=0.32\linewidth,height=\textheight,keepaspectratio]{results_excel/plots/Dane_TSP_127/p_make_grandchild/mean_std.png}
\includegraphics[width=0.32\linewidth,height=\textheight,keepaspectratio]{results_excel/plots/Dane_TSP_127/p_make_grandchild/time_vs_best_scatter.png}

Wraz ze wzrostem p\_make\_grandchild poprawiała się jakość rozwiązań do
wartości około 0.5. Powyżej tej granicy efekty się stabilizowały, ale
czas obliczeń wzrósł. Dla p\_make\_grandchild = 0.4-0.5 wyniki były
najrówniejsze i najczęściej dawały najkrótsze trasy.

\subsubsection{parametr
local\_search\_tries}\label{parametr-local_search_tries-2}

\includegraphics[width=0.32\linewidth,height=\textheight,keepaspectratio]{results_excel/plots/Dane_TSP_127/local_search_tries/boxplot_best_length.png}
\includegraphics[width=0.32\linewidth,height=\textheight,keepaspectratio]{results_excel/plots/Dane_TSP_127/local_search_tries/mean_std.png}
\includegraphics[width=0.32\linewidth,height=\textheight,keepaspectratio]{results_excel/plots/Dane_TSP_127/local_search_tries/time_vs_best_scatter.png}

Tutaj efekt jest jeszcze bardziej widoczny, brak lokalnego ulepszania
daje najgorsze wyniki (średnio \textasciitilde340-360k), a włączenie 5
prób obniża średnią do ok. 150k, natomiast dalsze zwiększanie do 10 i 20
prób przynosi stopniowe, ale widoczne dalsze usprawnienia. Najlepsze t
rasy uzyskuje się kosztem najdłuższego czasu.

\subsubsection{parametr
crossover\_method}\label{parametr-crossover_method-2}

\includegraphics[width=0.32\linewidth,height=\textheight,keepaspectratio]{results_excel/plots/Dane_TSP_127/crossover_method/boxplot_best_length.png}\includegraphics[width=0.32\linewidth,height=\textheight,keepaspectratio]{results_excel/plots/Dane_TSP_127/crossover_method/mean_std.png}
\includegraphics[width=0.32\linewidth,height=\textheight,keepaspectratio]{results_excel/plots/Dane_TSP_127/crossover_method/time_vs_best_scatter.png}

Krzyżowanie PMX daje najlepsze wyniki.

\subsubsection{parametr
selection\_method}\label{parametr-selection_method-2}

\includegraphics[width=0.32\linewidth,height=\textheight,keepaspectratio]{results_excel/plots/Dane_TSP_127/selection_method/boxplot_best_length.png}\includegraphics[width=0.32\linewidth,height=\textheight,keepaspectratio]{results_excel/plots/Dane_TSP_127/selection_method/mean_std.png}
\includegraphics[width=0.32\linewidth,height=\textheight,keepaspectratio]{results_excel/plots/Dane_TSP_127/selection_method/time_vs_best_scatter.png}

I jak w każdym z wcześniejszych przypadków, to selekcja poprzez turniej
wypada najlepiej.

\section{Podsumowanie i najlepszy zestaw
parametrów}\label{podsumowanie-i-najlepszy-zestaw-parametruxf3w}

\subsection{48 miast}\label{miast}

Na podstawie przeprowadzonej analizy dla problemu TSP z 48 miastami,
dostajemy najlepszy wynik (best\_length): 10628. To dzięki konfiguracji:

\begin{itemize}
\item
  n\_pop = 200
\item
  n\_gen = 300
\item
  p\_mut = 0.03
\item
  p\_cx = 0.9
\item
  selection\_method = tournament
\item
  crossover\_method = pmx
\item
  local\_search\_tries = 5
\item
  elite = 3
\item
  p\_make\_grandchild = 0
\end{itemize}

Najlepszy wynik uzyskaliśmy przy umiarkowanie dużej populacji i
standardowej liczbie generacji. Konfiguracja ta potwierdza skuteczność
parametrów bazowych, zwłaszcza włączenia algorytmu local\_search\_tries
i elitaryzmu. Wyłączenie mechanizmu „wnuka'' (p\_make\_grandchild = 0)
okazało się korzystne.

\subsection{76 miast}\label{miast-1}

Na podstawie przeprowadzonej analizy dla problemu TSP z 76 miastami,
dostajemy najlepszy wynik (best\_length): 108880. To dzięki
konfiguracji:

\begin{itemize}
\item
  n\_pop = 200
\item
  n\_gen = 300
\item
  p\_mut = 0.03
\item
  p\_cx = 0.9
\item
  selection\_method = tournament
\item
  crossover\_method = pmx
\item
  local\_search\_tries = 5
\item
  elite = 3
\item
  p\_make\_grandchild = 0
\end{itemize}

Dla instancji średniej wielkości, najlepsza konfiguracja okazała się
identyczna jak dla problemu 48 miast. Potwierdza to wysoką stabilność i
skuteczność tej kombinacji parametrów dla problemów tej skali.

\subsection{127 miast}\label{miast-2}

Na podstawie przeprowadzonej analizy dla problemu TSP ze 127 miastami,
dostajemy najlepszy wynik (best\_length): 120101. To dzięki
konfiguracji:

\begin{itemize}
\item
  n\_pop = 100
\item
  n\_gen = 600
\item
  p\_mut = 0.03
\item
  p\_cx = 0.9
\item
  selection\_method = tournament
\item
  crossover\_method = pmx
\item
  local\_search\_tries = 5
\item
  elite = 3
\item
  p\_make\_grandchild = 0
\end{itemize}

Dla największego problemu optymalna strategia uległa zmianie, algorytm
potrzebował mniejszej populacji, ale dwa razy dłuższej ewolucji.
Wskazuje to, że dla bardziej złożonych problemów kluczowe znaczenie ma
większa liczba iteracji i dłuższy czas na zbieganie. Warto jednak
zauważyć, że wszystkie pozostałe kluczowe parametry (mutacja,
krzyżowanie, selekcja, local search, elita i ``wnuk'') pozostały
identyczne jak w mniejszych instancjach.

\subsection{Wnioski}\label{wnioski}

Analizując konfiguracje, które dały absolutnie najlepsze wyniki dla
każdej z trzech instancji problemu, można zauważyć zaskakującą spójność.
Aż 7 z 9 testowanych grup parametrów było identycznych dla najlepszych
przebiegów:

\begin{itemize}
\item
  p\_mut = 0.03
\item
  p\_cx = 0.9
\item
  selection\_method = tournament
\item
  crossover\_method = pmx
\item
  local\_search\_tries = 5
\item
  elite = 3
\item
  p\_make\_grandchild = 0
\end{itemize}

Taka konfiguracja okazała się najbardziej uniwersalna. Nasze
usprawnienie - mechanizm „wnuka'' w żadnym z najlepszych przypadków nie
przyczynił się do poprawy wyniku.

\end{document}
