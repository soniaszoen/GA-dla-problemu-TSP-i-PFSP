% Options for packages loaded elsewhere
\PassOptionsToPackage{unicode}{hyperref}
\PassOptionsToPackage{hyphens}{url}
%
\documentclass[
]{article}
\usepackage{amsmath,amssymb}
\usepackage{iftex}
\ifPDFTeX
  \usepackage[T1]{fontenc}
  \usepackage[utf8]{inputenc}
  \usepackage{textcomp} % provide euro and other symbols
\else % if luatex or xetex
  \usepackage{unicode-math} % this also loads fontspec
  \defaultfontfeatures{Scale=MatchLowercase}
  \defaultfontfeatures[\rmfamily]{Ligatures=TeX,Scale=1}
\fi
\usepackage{lmodern}
\ifPDFTeX\else
  % xetex/luatex font selection
\fi
% Use upquote if available, for straight quotes in verbatim environments
\IfFileExists{upquote.sty}{\usepackage{upquote}}{}
\IfFileExists{microtype.sty}{% use microtype if available
  \usepackage[]{microtype}
  \UseMicrotypeSet[protrusion]{basicmath} % disable protrusion for tt fonts
}{}
\makeatletter
\@ifundefined{KOMAClassName}{% if non-KOMA class
  \IfFileExists{parskip.sty}{%
    \usepackage{parskip}
  }{% else
    \setlength{\parindent}{0pt}
    \setlength{\parskip}{6pt plus 2pt minus 1pt}}
}{% if KOMA class
  \KOMAoptions{parskip=half}}
\makeatother
\usepackage{xcolor}
\usepackage[margin=1in]{geometry}
\usepackage{graphicx}
\makeatletter
\newsavebox\pandoc@box
\newcommand*\pandocbounded[1]{% scales image to fit in text height/width
  \sbox\pandoc@box{#1}%
  \Gscale@div\@tempa{\textheight}{\dimexpr\ht\pandoc@box+\dp\pandoc@box\relax}%
  \Gscale@div\@tempb{\linewidth}{\wd\pandoc@box}%
  \ifdim\@tempb\p@<\@tempa\p@\let\@tempa\@tempb\fi% select the smaller of both
  \ifdim\@tempa\p@<\p@\scalebox{\@tempa}{\usebox\pandoc@box}%
  \else\usebox{\pandoc@box}%
  \fi%
}
% Set default figure placement to htbp
\def\fps@figure{htbp}
\makeatother
\setlength{\emergencystretch}{3em} % prevent overfull lines
\providecommand{\tightlist}{%
  \setlength{\itemsep}{0pt}\setlength{\parskip}{0pt}}
\setcounter{secnumdepth}{-\maxdimen} % remove section numbering
\usepackage{bookmark}
\IfFileExists{xurl.sty}{\usepackage{xurl}}{} % add URL line breaks if available
\urlstyle{same}
\hypersetup{
  pdftitle={Algorytm genetyczny dla permutacyjnego problemu przepływowego (PFSP)},
  hidelinks,
  pdfcreator={LaTeX via pandoc}}

\title{Algorytm genetyczny dla permutacyjnego problemu przepływowego
(PFSP)}
\author{}
\date{\vspace{-2.5em}2025-10-24}

\begin{document}
\maketitle

\section{Problem Przepływowy (PFSP)}\label{problem-przepux142ywowy-pfsp}

Rozważany problem to \textbf{Permutacyjny Problem Przepływowy}
(Permutation Flow Shop Scheduling Problem, PFSP). Celem jest znalezienie
optymalnej sekwencji (permutacji) \texttt{N} zadań, które muszą być
przetworzone na \texttt{M} maszynach w tej samej kolejności. Kryterium
optymalizacji jest minimalizacja czasu zakończenia ostatniego zadania na
ostatniej maszynie, znanego jako \textbf{makespan}.

\subsection{Założenia dotyczące problemu
PFSP}\label{zaux142oux17cenia-dotyczux105ce-problemu-pfsp}

\begin{itemize}
\tightlist
\item
  Wszystkie zadania (\texttt{N}) są dostępne do przetwarzania od samego
  początku.
\item
  Każde zadanie musi przejść przez wszystkie maszyny (\texttt{M}) w tej
  samej, z góry ustalonej kolejności.
\item
  Kolejność przetwarzania zadań jest identyczna na każdej maszynie (stąd
  nazwa ``permutacyjny'').
\item
  Czasy przetwarzania każdego zadania na każdej maszynie są znane i
  nieujemne.
\item
  Operacje nie mogą być przerywane.
\item
  Każda maszyna może w danym momencie przetwarzać tylko jedno zadanie.
\end{itemize}

\subsection{Funkcja celu}\label{funkcja-celu}

Główną metryką oceny jakości rozwiązania jest \textbf{makespan}. Jest to
czas, w którym zakończy się przetwarzanie ostatniego zadania w sekwencji
na ostatniej maszynie. Im mniejszy makespan, tym lepsze rozwiązanie.

\subsection{Dane}\label{dane}

Plik wejściowy zawiera macierz czasów przetwarzania zadań na maszynach:

\begin{itemize}
\tightlist
\item
  \texttt{Dane\_PFSP\_50\_20.xlsx} - instancja z 50 zadaniami i 20
  maszynami.
\item
  \texttt{Dane\_PFSP\_200\_10.xlsx} - instancja z 200 zadaniami i 10
  maszynami.
\item
  \texttt{Dane\_PFSP\_100\_10.xlsx} - instancja z 100 zadaniami i 10
  maszynami.
\end{itemize}

Algorytm wykorzystuje tę macierz do oceny (obliczenia makespanu) każdej
permutacji.

\section{Wstęp}\label{wstux119p}

Badamy algorytm genetyczny (GA rozwiązujący problem PFSP. Celem jest
porównanie wpływu kluczowych parametrów algorytmu na jakość znalezionych
tras, przyjmując podejście one-factor-at-a-time (OFAT) czyli zmieniamy
kolejno po jednym parametrze, a pozostałe pozostają stałe
(\texttt{baseline}). Wyniki zapisujemy do pliku Excela i analizujemy
wykresy.

\subsection{Środowisko i powtarzalność
eksperymentów}\label{ux15brodowisko-i-powtarzalnoux15bux107-eksperymentuxf3w}

\begin{itemize}
\tightlist
\item
  \textbf{Globalny seed}: \texttt{SEED\ =\ 42} jest ustawiany na
  początku działania skryptu.
\item
  \textbf{Seed dla uruchomienia}: Dla każdego pojedynczego uruchomienia
  algorytmu, seed jest ustawiany deterministycznie
  (\texttt{SEED\ +\ run\_id}), co pozwala na dokładne odtworzenie
  każdego eksperymentu.
\end{itemize}

\subsection{Założenia dotyczące algorytmu
genetycznego}\label{zaux142oux17cenia-dotyczux105ce-algorytmu-genetycznego}

Algorytm genetyczny jest metodą metaheurystyczną czyli nie gwarantuje
znalezienia rozwiązania optymalnego, ale pozwala znaleźć rozwiązania
bliskie optymalnym w rozsądnym czasie.

Rozwiązania (osobniki) są reprezentowane jako permutacje zadań -
kolejność wykonania zadań określa sekwencję. Populacja osobników
ewoluuje w kierunku coraz niższych wartości makespan.

\section{Opis parametrów}\label{opis-parametruxf3w}

\begin{itemize}
\item
  n\_pop oznacza rozmiar populacji; populacja początkowa jest tworzona
  losowo, a potem każda generacja jest w pełni zastępowana przez
  potomków + ewentualnie elity.
\item
  n\_gen oznacza liczbę generacji; im więcej generacji, tym dłużej trwa
  ewolucja i algorytm ma więcej szans na poprawę wyniku, ale zbyt wiele
  generacji może nie przynosić dalszych korzyści.
\item
  p\_mut czyli prawdopodobieństwo mutacji, pomaga uciekać z lokalnych
  minimów; szansa, że po utworzeniu dziecka, jego permutacja zostanie
  losowo zmieniona; w tym przypadku polega na zamianie (swap) dwóch
  zadań w kolejności; nie gwarantuje poprawy rozwiązania; (!nie należy
  mylić mutacji z ulepszaniem local search (ruchami sąsiedzkimi), które
  działają już po mutacji i próbują celowo poprawić permutację (obniżyć
  jej makespan), a nie losowo ją zmienić).
\item
  p\_cx czyli prawdopodobieństwo krzyżowania, częstotliwość rekombinacji
  rodziców; to szansa, że z dwóch rodziców powstanie nowy potomek, który
  łączy ich cechy; jeśli losowanie się nie powiedzie, potomek jest po
  prostu kopią jednego rodzica.
\item
  selection\_method czyli metody selekcji; decyduje kto zostanie wybrany
  jako rodzic do stworzenia potomka, wszystkie metody dążą do tego, by
  lepsi (osobnicy o niższym makespan) mieli większą szansę na
  rozmnożenie, ale każda robi to trochę inaczej:
\end{itemize}

\begin{enumerate}
\def\labelenumi{\arabic{enumi})}
\item
  tournament (selekcja turniejowa) - losuje k osobników z populacji, z
  tych k wybiera tego o najniższym makespanie (najlepszy), pozwala
  kontrolować siłę selekcji przez tournament\_k; im większe k, tym
  silniejsza selekcja.
\item
  roulette (selekcja ruletkowa) - każdy osobnik dostaje „wagę'' na
  ruletce proporcjonalną do swojej jakości (u nas odwrotnie do wartości
  makespan); daje większe szanse dobrym permutacjom, ale też pozwala
  czasem „słabszym'' wejść do gry (bo utrzymuje różnorodność).
\item
  rank (selekcja rankingowa) - najlepszy osobnik dostaje najwyższy
  „ranking'', najgorszy najniższy; szansa na wybór zależy od pozycji w
  rankingu, a nie bezpośrednio od wartości makespan. Działa stabilniej
  niż ruletka, gdy np. jeden osobnik jest drastycznie lepszy od reszty.
\end{enumerate}

\begin{itemize}
\tightlist
\item
  crossover\_method czyli metody krzyżowania, różne sposoby zachowania
  porządku i fragmentów rodziców w permutacji, czyli w jaki sposób
  łączone są ich geny (kolejności zadań):
\end{itemize}

\begin{enumerate}
\def\labelenumi{\arabic{enumi})}
\item
  PMX (Partially Mapped Crossover) - wybierany jest losowy fragment (np.
  4-8 zadań) z pierwszego rodzica i kopiowany do dziecka, reszta zadań
  uzupełniana jest na podstawie drugiego rodzica, zgodnie z mapowaniem
  między fragmentami.
\item
  OX (Order Crossover) - fragment z jednego rodzica zostaje zachowany,
  pozostałe zadania są wypełniane w kolejności ich występowania w drugim
  rodzicu; dziecko dziedziczy „kolejność względną'', a niekoniecznie
  dokładne pozycje.
\item
  CX (Cycle Crossover) - tworzy cykle powiązań między genami rodziców
  (np. zadanie A u pierwszego jest w tym samym miejscu co zadanie B u
  drugiego), naprzemiennie bierze cykle z rodzica 1 i 2.
\end{enumerate}

\begin{itemize}
\tightlist
\item
  local\_search\_tries: liczba prób lokalnego ulepszenia; czyli liczba
  ruchów sąsiedzkich, które są losowo testowane na każdym dziecku, by
  sprawdzić, czy da się poprawić makespan permutacji:
\end{itemize}

\begin{enumerate}
\def\labelenumi{\arabic{enumi})}
\item
  swap - zamiana dwóch zadań miejscami,
\item
  2-opt - odwrócenie fragmentu permutacji zadań,
\item
  insertion - wyjęcie zadania i wstawienie go gdzie indziej.
\end{enumerate}

! Uwaga ! Zamiast testować ruchy swap, insertion i two\_opt osobno,
funkcja local\_improve losowo wybiera jeden z nich w każdej próbie. Taki
model pozwala zbadać kluczową różnicę wpływu włączenia lokalnej poprawy
(nawiązanie do Algorytmu Memetycznego) w porównaniu do jej braku (co
stanowi czysty Algorytm Genetyczny).

\begin{itemize}
\tightlist
\item
  elite (elitaryzm) oznacza liczbę najlepszych osobników (o najniższym
  makespan) kopiowanych bez zmian do następnej generacji; chroni
  najlepsze znalezione rozwiązania, ale zbyt duża wartość elite mogłaby
  zmniejszyć różnorodność populacji.
\end{itemize}

\section{Usprawnienie czyli mechanizmu „Wnuka''
(grandchild)}\label{usprawnienie-czyli-mechanizmu-wnuka-grandchild}

Mechanizm „wnuka'' (grandchild) to proste rozszerzenie standardowego
schematu rozmnażania. Po wygenerowaniu potomka (child) istnieje pewne
prawdopodobieństwo p\_make\_grandchild, że z potomka powstanie dodatkowy
twór - „wnuk'' (grandchild). Wnuk powstaje przez zastosowanie kolejnego,
krótkiego operatora naprawczego (np. prostej mutacji lub szybkiego
lokalnego ulepszania). Do populacji przyjmowany jest lepszy z pary
(child vs grandchild), czyli ten o niższym makespanie. Intuicyjnie,
zamiast przyjmować bezwarunkowo pojedynczego potomka, sprawdzamy jedną
dodatkową opcję „bliźniaczą'' i zachowujemy tę, która daje lepszy wynik.

Parametry:

\begin{itemize}
\item
  p\_make\_grandchild (= {[}0,1{]}) - prawdopodobieństwo wygenerowania
  wnuka dla każdego potomka.
\item
  grandchild\_method = \{`mutate', `local\_improve'\} - sposób
  wytworzenia wnuka: prosta mutacja (swap) lub krótki lokalny search
  (wykorzystujący tę samą funkcję local\_improve, co główna pętla).
\item
  grandchild\_local\_tries - liczba prób dla lokalnego ulepszania wnuka
  (jeśli None, używany jest ogólny parametr local\_search\_tries).
\end{itemize}

Mechanizm wnuka daje prosty kompromis:

\begin{itemize}
\item
  zwiększa szansę na znalezienie natychmiastowej ulepszonej wersji
  potomka (przy niskim koszcie - jedno wywołanie mutacji lub krótkiego
  local search);
\item
  jest prosty do kontrolowania przez pojedynczy parametr
  p\_make\_grandchild.
\end{itemize}

Efekty obserwowane w eksperymentach:

\begin{itemize}
\item
  umiarkowane zwiększenie jakości końcowych rozwiązań (niższe średnie i
  lepsze mediany),
\item
  niewielki wzrost wariancji (jeśli p\_make\_grandchild zbyt duże),
\end{itemize}

\section{Konfiguracja eksperymentu
(OFAT)}\label{konfiguracja-eksperymentu-ofat}

\subsection{Parametry domyślne
(Baseline)}\label{parametry-domyux15blne-baseline}

Konfiguracja bazowa, od której wychodzimy przy testowaniu każdego
parametru:

\begin{itemize}
\tightlist
\item
  \texttt{n\_pop}: 100,
\item
  \texttt{n\_gen}: 100,
\item
  \texttt{p\_mut}: 0.03,
\item
  \texttt{p\_cx}: 0.9,
\item
  \texttt{selection\_method}: `tournament',
\item
  \texttt{crossover\_method}: `pmx',
\item
  \texttt{local\_search\_tries}: 3,
\item
  \texttt{local\_moves}: {[}`swap',`two\_opt',`insertion'{]},
\item
  \texttt{elite}: 3,
\item
  \texttt{tournament\_k}: 3,
\item
  \texttt{p\_make\_grandchild}: 0.0,
\item
  \texttt{grandchild\_method}: `local\_improve',
\item
  \texttt{grandchild\_local\_tries}: 3
\end{itemize}

\subsection{Parametry testowane}\label{parametry-testowane}

Dla każdego z poniższych parametrów testowano podane wartości, podczas
gdy reszta pozostawała zgodna z konfiguracją \texttt{baseline}. Każda
konfiguracja została uruchomiona \textbf{10 razy}.

\begin{itemize}
\tightlist
\item
  \texttt{n\_pop}: {[}50, 100, 200, 400{]}
\item
  \texttt{n\_gen}: {[}100, 300, 600, 1000{]}
\item
  \texttt{p\_mut}: {[}0.01, 0.03, 0.07, 0.15{]}
\item
  \texttt{p\_cx}: {[}0.6, 0.75, 0.9, 1.0{]}
\item
  \texttt{elite}: {[}1, 3, 5, 10{]}
\item
  \texttt{local\_search\_tries}: {[}0, 3, 7, 15{]}
\item
  \texttt{p\_make\_grandchild}: {[}0.0, 0.1, 0.3, 0.5{]}
\item
  \texttt{crossover\_method}: {[}`pmx', `ox', `cx'{]},
\item
  \texttt{selection\_method}: {[}`tournament', `roulette', `rank'{]}
\end{itemize}

\section{Analiza wyników}\label{analiza-wynikuxf3w}

Dla każdego parametru przedstawiono:

\begin{enumerate}
\def\labelenumi{\arabic{enumi}.}
\item
  \textbf{Wykres pudełkowy (boxplot)} - pokazuje rozkład, medianę i
  wartości odstające (outliery) końcowego makespanu.
\item
  \textbf{Wykres średniej i odchylenia standardowego} - obrazuje średnią
  jakość i stabilność wyników.
\item
  \textbf{Wykres rozrzutu (scatter plot)} - przedstawia zależność między
  czasem obliczeń a osiągniętym makespanem.
\end{enumerate}

\subsection{Dane PFSP\_50\_20}\label{dane-pfsp_50_20}

\subsubsection{\texorpdfstring{parametr \texttt{n\_pop} (wielkość
populacji)}{parametr n\_pop (wielkość populacji)}}\label{parametr-n_pop-wielkoux15bux107-populacji}

\includegraphics[width=0.32\linewidth,height=\textheight,keepaspectratio]{results_excel_pfsp/plots/PFSP_50_20/n_pop/boxplot_best_makespan.png}
\includegraphics[width=0.32\linewidth,height=\textheight,keepaspectratio]{results_excel_pfsp/plots/PFSP_50_20/n_pop/mean_std.png}
\includegraphics[width=0.32\linewidth,height=\textheight,keepaspectratio]{results_excel_pfsp/plots/PFSP_50_20/n_pop/time_vs_best_scatter.png}
Zwiększanie wielkości populacji z 50 do 200 przynosi wyraźną poprawę
zarówno średniego makespanu, jak i stabilności wyników (mniejszy
rozrzut). Dalsze zwiększanie populacji do 400 nie daje już znaczącej
poprawy jakości, natomiast znacząco wydłuża czas obliczeń, co widać na
wykresie rozrzutu. Optymalny kompromis między jakością rozwiązania a
czasem działania dla tej instancji problemu stanowi populacja o
wielkości \textbf{200 osobników}.

\subsubsection{\texorpdfstring{Parametr \texttt{n\_gen} (liczba
generacji)}{Parametr n\_gen (liczba generacji)}}\label{parametr-n_gen-liczba-generacji}

\includegraphics[width=0.32\linewidth,height=\textheight,keepaspectratio]{results_excel_pfsp/plots/PFSP_50_20/n_gen/boxplot_best_makespan.png}
\includegraphics[width=0.32\linewidth,height=\textheight,keepaspectratio]{results_excel_pfsp/plots/PFSP_50_20/n_gen/mean_std.png}
\includegraphics[width=0.32\linewidth,height=\textheight,keepaspectratio]{results_excel_pfsp/plots/PFSP_50_20/n_gen/time_vs_best_scatter.png}

Wraz ze wzrostem liczby generacji, średni makespan maleje, co jest
oczekiwanym zachowaniem. Wartość \textbf{300-600 generacji} wydaje się
być najlepszym wyborem, zapewniając dobre wyniki w akceptowalnym czasie.

\subsubsection{\texorpdfstring{parametr \texttt{p\_mut}
(prawdopodobieństwo
mutacji)}{parametr p\_mut (prawdopodobieństwo mutacji)}}\label{parametr-p_mut-prawdopodobieux144stwo-mutacji}

\includegraphics[width=0.32\linewidth,height=\textheight,keepaspectratio]{results_excel_pfsp/plots/PFSP_50_20/p_mut/boxplot_best_makespan.png}
\includegraphics[width=0.32\linewidth,height=\textheight,keepaspectratio]{results_excel_pfsp/plots/PFSP_50_20/p_mut/mean_std.png}
\includegraphics[width=0.32\linewidth,height=\textheight,keepaspectratio]{results_excel_pfsp/plots/PFSP_50_20/p_mut/time_vs_best_scatter.png}

Zbyt duża mutacja pogarsza stabilność. Optymalne wartość to 0,03.

\subsubsection{\texorpdfstring{parametr \texttt{p\_cx}
(prawdopodobieństwo
krzyżowania)}{parametr p\_cx (prawdopodobieństwo krzyżowania)}}\label{parametr-p_cx-prawdopodobieux144stwo-krzyux17cowania}

\includegraphics[width=0.32\linewidth,height=\textheight,keepaspectratio]{results_excel_pfsp/plots/PFSP_50_20/p_cx/boxplot_best_makespan.png}
\includegraphics[width=0.32\linewidth,height=\textheight,keepaspectratio]{results_excel_pfsp/plots/PFSP_50_20/p_cx/mean_std.png}
\includegraphics[width=0.32\linewidth,height=\textheight,keepaspectratio]{results_excel_pfsp/plots/PFSP_50_20/p_cx/time_vs_best_scatter.png}
Najlepsze wyniki pomiędzy 0,6-0,75.

\subsubsection{\texorpdfstring{parametr \texttt{elite} (liczba osobników
elitarnych)}{parametr elite (liczba osobników elitarnych)}}\label{parametr-elite-liczba-osobnikuxf3w-elitarnych}

\includegraphics[width=0.32\linewidth,height=\textheight,keepaspectratio]{results_excel_pfsp/plots/PFSP_50_20/elite/boxplot_best_makespan.png}
\includegraphics[width=0.32\linewidth,height=\textheight,keepaspectratio]{results_excel_pfsp/plots/PFSP_50_20/elite/mean_std.png}
\includegraphics[width=0.32\linewidth,height=\textheight,keepaspectratio]{results_excel_pfsp/plots/PFSP_50_20/elite/time_vs_best_scatter.png}

Zwiększenie elity do 3-5 osobników stabilizuje wyniki i chroni najlepsze
rozwiązania, bez utraty różnorodności. Powyżej 5 efekt jest marginalny.

\subsubsection{\texorpdfstring{parametr \texttt{local\_search\_tries}
(liczba prób ulepszenia
lokalnego)}{parametr local\_search\_tries (liczba prób ulepszenia lokalnego)}}\label{parametr-local_search_tries-liczba-pruxf3b-ulepszenia-lokalnego}

\includegraphics[width=0.32\linewidth,height=\textheight,keepaspectratio]{results_excel_pfsp/plots/PFSP_50_20/local_search_tries/boxplot_best_makespan.png}\includegraphics[width=0.32\linewidth,height=\textheight,keepaspectratio]{results_excel_pfsp/plots/PFSP_50_20/local_search_tries/mean_std.png}
\includegraphics[width=0.32\linewidth,height=\textheight,keepaspectratio]{results_excel_pfsp/plots/PFSP_50_20/local_search_tries/time_vs_best_scatter.png}

Wyłączenie mechanizmu lokalnego poszukiwania (\texttt{tries\ =\ 0}) daje
zdecydowanie najgorsze wyniki. Już włączenie 3 prób powoduje drastyczną
poprawę makespanu. Dalsze zwiększanie liczby prób (do 7 i 15) przynosi
dodatkową, choć już mniejszą, poprawę. Wykres czasu do jakości pokazuje,
że lepsze wyniki osiągane są kosztem dłuższego czasu działania.
Mechanizm lokalnego ulepszenia jest \textbf{kluczowy dla jakości}.
Wartość w zakresie \textbf{3-7 prób} stanowi dobry kompromis.

\subsubsection{\texorpdfstring{parametr \texttt{p\_make\_grandchild}
(prawdopodobieństwo
``wnuka'')}{parametr p\_make\_grandchild (prawdopodobieństwo ``wnuka'')}}\label{parametr-p_make_grandchild-prawdopodobieux144stwo-wnuka}

\includegraphics[width=0.32\linewidth,height=\textheight,keepaspectratio]{results_excel_pfsp/plots/PFSP_50_20/p_make_grandchild/boxplot_best_makespan.png}
\includegraphics[width=0.32\linewidth,height=\textheight,keepaspectratio]{results_excel_pfsp/plots/PFSP_50_20/p_make_grandchild/mean_std.png}
\includegraphics[width=0.32\linewidth,height=\textheight,keepaspectratio]{results_excel_pfsp/plots/PFSP_50_20/p_make_grandchild/time_vs_best_scatter.png}
Wraz ze wzrostem prawdopodobieństwa p\_make\_grandchild obserwuje się
umiarkowaną poprawę jakości rozwiązań. Przy 0.5 jakość przestaje się
poprawiać. Optymalna wartość znajduje się pomiędzy 0,1-0,2.

\subsubsection{parametr
crossover\_method}\label{parametr-crossover_method}

\includegraphics[width=0.32\linewidth,height=\textheight,keepaspectratio]{results_excel_pfsp/plots/PFSP_50_20/crossover_method/boxplot_best_makespan.png}
\includegraphics[width=0.32\linewidth,height=\textheight,keepaspectratio]{results_excel_pfsp/plots/PFSP_50_20/crossover_method/mean_std.png}
\includegraphics[width=0.32\linewidth,height=\textheight,keepaspectratio]{results_excel_pfsp/plots/PFSP_50_20/crossover_method/time_vs_best_scatter.png}

Metody PMX i OX zapewniają podobną jakość, ale PMX daje stabilniejsze
wyniki. To PMX wykorzystujemy jako podstawową metodę krzyżowania.

\subsubsection{parametr
selection\_method}\label{parametr-selection_method}

\includegraphics[width=0.3\linewidth,height=\textheight,keepaspectratio]{results_excel_pfsp/plots/PFSP_50_20/selection_method/boxplot_best_makespan.png}
\includegraphics[width=0.32\linewidth,height=\textheight,keepaspectratio]{results_excel_pfsp/plots/PFSP_50_20/selection_method/mean_std.png}
\includegraphics[width=0.3\linewidth,height=\textheight,keepaspectratio]{results_excel_pfsp/plots/PFSP_50_20/selection_method/time_vs_best_scatter.png}
Selekcja turniejowa jest najbardziej niezawodna, zapewnia szybkie
zbieganie i stabilność wyników.

\subsection{Dane PFSP\_200\_10}\label{dane-pfsp_200_10}

\subsubsection{\texorpdfstring{parametr \texttt{n\_pop} (wielkość
populacji)}{parametr n\_pop (wielkość populacji)}}\label{parametr-n_pop-wielkoux15bux107-populacji-1}

\includegraphics[width=0.32\linewidth,height=\textheight,keepaspectratio]{results_excel_pfsp/plots/PFSP_200_10/n_pop/boxplot_best_makespan.png}
\includegraphics[width=0.32\linewidth,height=\textheight,keepaspectratio]{results_excel_pfsp/plots/PFSP_200_10/n_pop/mean_std.png}
\includegraphics[width=0.32\linewidth,height=\textheight,keepaspectratio]{results_excel_pfsp/plots/PFSP_200_10/n_pop/time_vs_best_scatter.png}
Wraz ze wzrostem populacji poprawia się jakość rozwiązań, ale przy
znacznym wzroście czasu. Populacja 200-400 osobników to najlepszy
kompromis.

\subsubsection{\texorpdfstring{parametr \texttt{n\_gen} (liczba
generacji)}{parametr n\_gen (liczba generacji)}}\label{parametr-n_gen-liczba-generacji-1}

\includegraphics[width=0.32\linewidth,height=\textheight,keepaspectratio]{results_excel_pfsp/plots/PFSP_200_10/n_gen/boxplot_best_makespan.png}
\includegraphics[width=0.32\linewidth,height=\textheight,keepaspectratio]{results_excel_pfsp/plots/PFSP_200_10/n_gen/mean_std.png}
\includegraphics[width=0.32\linewidth,height=\textheight,keepaspectratio]{results_excel_pfsp/plots/PFSP_200_10/n_gen/time_vs_best_scatter.png}

Większa liczba generacji przynosi korzyści aż do około 600, po czym zysk
jakościowy maleje.

\subsubsection{\texorpdfstring{parametr \texttt{p\_mut}
(prawdopodobieństwo
mutacji)}{parametr p\_mut (prawdopodobieństwo mutacji)}}\label{parametr-p_mut-prawdopodobieux144stwo-mutacji-1}

\includegraphics[width=0.32\linewidth,height=\textheight,keepaspectratio]{results_excel_pfsp/plots/PFSP_200_10/p_mut/boxplot_best_makespan.png}
\includegraphics[width=0.32\linewidth,height=\textheight,keepaspectratio]{results_excel_pfsp/plots/PFSP_200_10/p_mut/mean_std.png}
\includegraphics[width=0.32\linewidth,height=\textheight,keepaspectratio]{results_excel_pfsp/plots/PFSP_200_10/p_mut/time_vs_best_scatter.png}

Wartość 0,07 daje najlepszy efekt.

\subsubsection{\texorpdfstring{parametr \texttt{p\_cx}
(prawdopodobieństwo
krzyżowania)}{parametr p\_cx (prawdopodobieństwo krzyżowania)}}\label{parametr-p_cx-prawdopodobieux144stwo-krzyux17cowania-1}

\includegraphics[width=0.32\linewidth,height=\textheight,keepaspectratio]{results_excel_pfsp/plots/PFSP_200_10/p_cx/boxplot_best_makespan.png}
\includegraphics[width=0.32\linewidth,height=\textheight,keepaspectratio]{results_excel_pfsp/plots/PFSP_200_10/p_cx/mean_std.png}
\includegraphics[width=0.32\linewidth,height=\textheight,keepaspectratio]{results_excel_pfsp/plots/PFSP_200_10/p_cx/time_vs_best_scatter.png}

Wysokie prawdopodobieństwo krzyżowania zwiększa szansę na dobre
kombinacje genów, 0,9 jest najbardziej efektywne.

\subsubsection{\texorpdfstring{parametr \texttt{elite} (liczba osobników
elitarnych)}{parametr elite (liczba osobników elitarnych)}}\label{parametr-elite-liczba-osobnikuxf3w-elitarnych-1}

\includegraphics[width=0.32\linewidth,height=\textheight,keepaspectratio]{results_excel_pfsp/plots/PFSP_200_10/elite/boxplot_best_makespan.png}
\includegraphics[width=0.32\linewidth,height=\textheight,keepaspectratio]{results_excel_pfsp/plots/PFSP_200_10/elite/mean_std.png}
\includegraphics[width=0.32\linewidth,height=\textheight,keepaspectratio]{results_excel_pfsp/plots/PFSP_200_10/elite/time_vs_best_scatter.png}

Umiarkowany elitaryzm pozwala zachować balans między ochroną najlepszych
a utrzymaniem różnorodności. Optymalnie 3 elity.

\subsubsection{\texorpdfstring{parametr \texttt{local\_search\_tries}
(liczba prób ulepszenia
lokalnego)}{parametr local\_search\_tries (liczba prób ulepszenia lokalnego)}}\label{parametr-local_search_tries-liczba-pruxf3b-ulepszenia-lokalnego-1}

\includegraphics[width=0.32\linewidth,height=\textheight,keepaspectratio]{results_excel_pfsp/plots/PFSP_200_10/local_search_tries/boxplot_best_makespan.png}\includegraphics[width=0.32\linewidth,height=\textheight,keepaspectratio]{results_excel_pfsp/plots/PFSP_200_10/local_search_tries/mean_std.png}\includegraphics[width=0.32\linewidth,height=\textheight,keepaspectratio]{results_excel_pfsp/plots/PFSP_200_10/local_search_tries/time_vs_best_scatter.png}

Większa liczba prób ulepszenia lokalnego poprawia wyniki, ale rośnie też
czas obliczeń. 7 prób to dobry kompromis.

\subsubsection{\texorpdfstring{parametr \texttt{p\_make\_grandchild}
(prawdopodobieństwo
``wnuka'')}{parametr p\_make\_grandchild (prawdopodobieństwo ``wnuka'')}}\label{parametr-p_make_grandchild-prawdopodobieux144stwo-wnuka-1}

\includegraphics[width=0.32\linewidth,height=\textheight,keepaspectratio]{results_excel_pfsp/plots/PFSP_200_10/p_make_grandchild/boxplot_best_makespan.png}
\includegraphics[width=0.32\linewidth,height=\textheight,keepaspectratio]{results_excel_pfsp/plots/PFSP_200_10/p_make_grandchild/mean_std.png}\includegraphics[width=0.32\linewidth,height=\textheight,keepaspectratio]{results_excel_pfsp/plots/PFSP_200_10/p_make_grandchild/time_vs_best_scatter.png}

Wartości około 0,3 przynoszą wyraźną poprawę jakości bez dużego wzrostu
czasu działania.

\subsubsection{\texorpdfstring{parametr \texttt{crossover\_method}
(metody
krzyżowania)}{parametr crossover\_method (metody krzyżowania)}}\label{parametr-crossover_method-metody-krzyux17cowania}

\includegraphics[width=0.32\linewidth,height=\textheight,keepaspectratio]{results_excel_pfsp/plots/PFSP_200_10/crossover_method/boxplot_best_makespan.png}\includegraphics[width=0.32\linewidth,height=\textheight,keepaspectratio]{results_excel_pfsp/plots/PFSP_200_10/crossover_method/mean_std.png}
\includegraphics[width=0.32\linewidth,height=\textheight,keepaspectratio]{results_excel_pfsp/plots/PFSP_200_10/crossover_method/time_vs_best_scatter.png}

PMX pozostaje najbardziej stabilny.

\subsubsection{\texorpdfstring{parametr \texttt{selection\_method}
(metody
selekcji)}{parametr selection\_method (metody selekcji)}}\label{parametr-selection_method-metody-selekcji}

\includegraphics[width=0.32\linewidth,height=\textheight,keepaspectratio]{results_excel_pfsp/plots/PFSP_200_10/selection_method/boxplot_best_makespan.png}\includegraphics[width=0.32\linewidth,height=\textheight,keepaspectratio]{results_excel_pfsp/plots/PFSP_200_10/selection_method/mean_std.png}\includegraphics[width=0.32\linewidth,height=\textheight,keepaspectratio]{results_excel_pfsp/plots/PFSP_200_10/selection_method/time_vs_best_scatter.png}

Selekcja turniejowa zapewnia najlepszy stosunek jakości do czasu.

\subsection{Dane PFSP\_100\_10}\label{dane-pfsp_100_10}

\subsubsection{\texorpdfstring{parametr \texttt{n\_pop} (wielkość
populacji)}{parametr n\_pop (wielkość populacji)}}\label{parametr-n_pop-wielkoux15bux107-populacji-2}

\includegraphics[width=0.32\linewidth,height=\textheight,keepaspectratio]{results_excel_pfsp/plots/PFSP_100_10/n_pop/boxplot_best_makespan.png}
\includegraphics[width=0.32\linewidth,height=\textheight,keepaspectratio]{results_excel_pfsp/plots/PFSP_100_10/n_pop/mean_std.png}
\includegraphics[width=0.32\linewidth,height=\textheight,keepaspectratio]{results_excel_pfsp/plots/PFSP_100_10/n_pop/time_vs_best_scatter.png}

Wielkość populacji 200 zapewnia najwyższą jakość przy rozsądnym czasie
działania.

\subsubsection{\texorpdfstring{parametr \texttt{n\_gen} (liczba
generacji)}{parametr n\_gen (liczba generacji)}}\label{parametr-n_gen-liczba-generacji-2}

\includegraphics[width=0.32\linewidth,height=\textheight,keepaspectratio]{results_excel_pfsp/plots/PFSP_100_10/n_gen/boxplot_best_makespan.png}
\includegraphics[width=0.32\linewidth,height=\textheight,keepaspectratio]{results_excel_pfsp/plots/PFSP_100_10/n_gen/mean_std.png}
\includegraphics[width=0.32\linewidth,height=\textheight,keepaspectratio]{results_excel_pfsp/plots/PFSP_100_10/n_gen/time_vs_best_scatter.png}

Wyniki poprawiają się wraz z generacjami do około 600, potem korzyści są
minimalne.

\subsubsection{\texorpdfstring{parametr \texttt{p\_mut}
(prawdopodobieństwo
mutacji)}{parametr p\_mut (prawdopodobieństwo mutacji)}}\label{parametr-p_mut-prawdopodobieux144stwo-mutacji-2}

\includegraphics[width=0.32\linewidth,height=\textheight,keepaspectratio]{results_excel_pfsp/plots/PFSP_100_10/p_mut/boxplot_best_makespan.png}
\includegraphics[width=0.32\linewidth,height=\textheight,keepaspectratio]{results_excel_pfsp/plots/PFSP_100_10/p_mut/mean_std.png}
\includegraphics[width=0.32\linewidth,height=\textheight,keepaspectratio]{results_excel_pfsp/plots/PFSP_100_10/p_mut/time_vs_best_scatter.png}

Najlepszą równowagę między różnorodnością a stabilnością zapewnia
wartość 0,07.

\subsubsection{\texorpdfstring{parametr \texttt{p\_cx}
(prawdopodobieństwo
krzyżowania)}{parametr p\_cx (prawdopodobieństwo krzyżowania)}}\label{parametr-p_cx-prawdopodobieux144stwo-krzyux17cowania-2}

\includegraphics[width=0.32\linewidth,height=\textheight,keepaspectratio]{results_excel_pfsp/plots/PFSP_100_10/p_cx/boxplot_best_makespan.png}
\includegraphics[width=0.32\linewidth,height=\textheight,keepaspectratio]{results_excel_pfsp/plots/PFSP_100_10/p_cx/mean_std.png}
\includegraphics[width=0.32\linewidth,height=\textheight,keepaspectratio]{results_excel_pfsp/plots/PFSP_100_10/p_cx/time_vs_best_scatter.png}

Wartość około 0,75 umożliwia najlepsze łączenie cech rodziców i stabilne
wyniki.

\subsubsection{\texorpdfstring{parametr \texttt{elite} (liczba osobników
elitarnych)}{parametr elite (liczba osobników elitarnych)}}\label{parametr-elite-liczba-osobnikuxf3w-elitarnych-2}

\includegraphics[width=0.32\linewidth,height=\textheight,keepaspectratio]{results_excel_pfsp/plots/PFSP_100_10/elite/boxplot_best_makespan.png}
\includegraphics[width=0.32\linewidth,height=\textheight,keepaspectratio]{results_excel_pfsp/plots/PFSP_100_10/elite/mean_std.png}
\includegraphics[width=0.32\linewidth,height=\textheight,keepaspectratio]{results_excel_pfsp/plots/PFSP_100_10/elite/time_vs_best_scatter.png}

Najlepszy efekt daje 3 elity, zapewniając stabilność i brak utraty
różnorodności.

\subsubsection{\texorpdfstring{parametr \texttt{local\_search\_tries}
(liczba prób ulepszenia
lokalnego)}{parametr local\_search\_tries (liczba prób ulepszenia lokalnego)}}\label{parametr-local_search_tries-liczba-pruxf3b-ulepszenia-lokalnego-2}

\includegraphics[width=0.32\linewidth,height=\textheight,keepaspectratio]{results_excel_pfsp/plots/PFSP_100_10/local_search_tries/boxplot_best_makespan.png}\includegraphics[width=0.32\linewidth,height=\textheight,keepaspectratio]{results_excel_pfsp/plots/PFSP_100_10/local_search_tries/mean_std.png}
\includegraphics[width=0.32\linewidth,height=\textheight,keepaspectratio]{results_excel_pfsp/plots/PFSP_100_10/local_search_tries/time_vs_best_scatter.png}

Brak lokalnego ulepszania znacząco pogarsza wyniki; 3-7 prób daje
wyraźną poprawę jakości.

\subsubsection{\texorpdfstring{parametr \texttt{p\_make\_grandchild}
(prawdopodobieństwo
``wnuka'')}{parametr p\_make\_grandchild (prawdopodobieństwo ``wnuka'')}}\label{parametr-p_make_grandchild-prawdopodobieux144stwo-wnuka-2}

\includegraphics[width=0.32\linewidth,height=\textheight,keepaspectratio]{results_excel_pfsp/plots/PFSP_100_10/p_make_grandchild/boxplot_best_makespan.png}\includegraphics[width=0.32\linewidth,height=\textheight,keepaspectratio]{results_excel_pfsp/plots/PFSP_100_10/p_make_grandchild/mean_std.png}\includegraphics[width=0.32\linewidth,height=\textheight,keepaspectratio]{results_excel_pfsp/plots/PFSP_100_10/p_make_grandchild/time_vs_best_scatter.png}

Najlepsze rezultaty uzyskuje się dla p\_make\_grandchild = 0,1, co
umiarkowanie poprawia jakość rozwiązań.

\subsubsection{\texorpdfstring{parametr \texttt{crossover\_method}
(metody
krzyżowania)}{parametr crossover\_method (metody krzyżowania)}}\label{parametr-crossover_method-metody-krzyux17cowania-1}

\includegraphics[width=0.32\linewidth,height=\textheight,keepaspectratio]{results_excel_pfsp/plots/PFSP_100_10/crossover_method/boxplot_best_makespan.png}\includegraphics[width=0.32\linewidth,height=\textheight,keepaspectratio]{results_excel_pfsp/plots/PFSP_100_10/crossover_method/mean_std.png}
\includegraphics[width=0.32\linewidth,height=\textheight,keepaspectratio]{results_excel_pfsp/plots/PFSP_100_10/crossover_method/time_vs_best_scatter.png}

PMX i CX wypadają podobnie, jednak PMX daje bardziej spójne wyniki.

\subsubsection{\texorpdfstring{parametr \texttt{selection\_method}
(metody
selekcji)}{parametr selection\_method (metody selekcji)}}\label{parametr-selection_method-metody-selekcji-1}

\includegraphics[width=0.32\linewidth,height=\textheight,keepaspectratio]{results_excel_pfsp/plots/PFSP_100_10/selection_method/boxplot_best_makespan.png}\includegraphics[width=0.32\linewidth,height=\textheight,keepaspectratio]{results_excel_pfsp/plots/PFSP_100_10/selection_method/mean_std.png}
\includegraphics[width=0.32\linewidth,height=\textheight,keepaspectratio]{results_excel_pfsp/plots/PFSP_100_10/selection_method/time_vs_best_scatter.png}

Turniejowa selekcja utrzymuje stabilność i szybkość.

\section{Podsumowanie i najlepszy zestaw
parametrów}\label{podsumowanie-i-najlepszy-zestaw-parametruxf3w}

\subsection{50 zadań i 20 maszyn}\label{zadaux144-i-20-maszyn}

Na podstawie przeprowadzonej analizy dla problemu PFSP z 50 zadaniami i
20 maszynami, dostajemy najlepszy wynik (best\_makespan): 3761. To
dzięki konfiguracji:

\begin{itemize}
\tightlist
\item
  n\_pop = 200\\
\item
  n\_gen = 100\\
\item
  p\_mut = 0.03\\
\item
  p\_cx = 0.9\\
\item
  selection\_method = tournament\\
\item
  crossover\_method = pmx\\
\item
  local\_search\_tries = 3\\
\item
  elite = 3\\
\item
  p\_make\_grandchild = 0
\end{itemize}

Najlepszy wynik uzyskaliśmy przy umiarkowanej populacji i niewielkiej
liczbie generacji, co sugeruje, że już w początkowych etapach algorytm
zbiega do wysokiej jakości rozwiązań. Optymalne okazały się umiarkowane
wartości mutacji i wysokie krzyżowanie, a wyłączenie mechanizmu
(rzekomego ulepszenia) „wnuka'' skróciło czas obliczeń bez pogorszenia
jakości. Konfiguracja ta jest najbardziej efektywna obliczeniowo.

\subsection{200 zadań i 10 maszyn}\label{zadaux144-i-10-maszyn}

Na podstawie przeprowadzonej analizy dla problemu PFSP składającego się
z 200 zadań i 20 maszyn, dostajemy najlepszy wynik (best\_makespan):
10592. To dzięki konfiguracji:

\begin{itemize}
\tightlist
\item
  n\_pop = 100\\
\item
  n\_gen = 1000\\
\item
  p\_mut = 0.03\\
\item
  p\_cx = 0.9\\
\item
  selection\_method = tournament\\
\item
  crossover\_method = pmx\\
\item
  local\_search\_tries = 3\\
\item
  elite = 3\\
\item
  p\_make\_grandchild = 0
\end{itemize}

Najlepszy wynik uzyskaliśmy przy dużej liczbie generacji i umiarkowanej
populacji, co wskazuje, że kluczowe znaczenie ma dłuższa ewolucja
populacji. Umiarkowane prawdopodobieństwo mutacji i wysokie krzyżowanie
pozwoliły zachować równowagę. Brak mechanizmu „wnuka'' przyspieszył
obliczenia bez utraty jakości, dzięki czemu konfiguracja ta jest
najbardziej skuteczna i stabilna dla dużej instancji.

\subsection{200 zadań i 10 maszyn}\label{zadaux144-i-10-maszyn-1}

Na podstawie przeprowadzonej analizy dla problemu PFSP składającego się
z 100 zadań i 10 maszyn, dostajemy najlepszy wynik (best\_makespan):
5516. To dzięki konfiguracji:

\begin{itemize}
\tightlist
\item
  n\_pop = 100\\
\item
  n\_gen = 1000\\
\item
  p\_mut = 0.03\\
\item
  p\_cx = 0.9\\
\item
  selection\_method = tournament\\
\item
  crossover\_method = pmx\\
\item
  local\_search\_tries = 3\\
\item
  elite = 3\\
\item
  p\_make\_grandchild = 0
\end{itemize}

Najlepszy wynik dostaliśmy przy dużej liczbie generacji i umiarkowanej
populacji. Stabilna konfiguracja z niskim prawdopodobieństwem mutacji i
wysokim krzyżowaniem pozwoliła na skuteczną eksplorację przestrzeni
rozwiązań. W każdym przypadku lepsze wyniki otrzymaliśmy bez mechanizmu
wnuka.

\subsection{Wnioski}\label{wnioski}

W każdym z trzech przypadków dla najlepszych wyników te parametry
pozostały niezmienne:

\begin{itemize}
\tightlist
\item
  p\_mut = 0.03\\
\item
  p\_cx = 0.9\\
\item
  selection\_method = tournament\\
\item
  crossover\_method = pmx\\
\item
  local\_search\_tries = 3\\
\item
  elite = 3\\
\item
  p\_make\_grandchild = 0
\end{itemize}

Taka konfiguracja okazała się najbardziej uniwersalna, zapewniając
jednocześnie wysoką jakość rozwiązań i stabilność działania algorytmu.

\end{document}
